\usepackage{tikz-timing}
\usetheme{CambridgeUS}
\usecolortheme{beaver}
\definecolor{darkred}{RGB}{122, 0, 25}

%\setbeamercolor{palette tertiary}{bg=maroon}
\setbeamercolor{titlelike}{bg=gray!15!white}
\setbeamertemplate{itemize items}[default]
\setbeamertemplate{itemize subitem}[circle]
\setbeamertemplate{enumerate items}[default]

\setbeamertemplate{blocks}[rounded]
\setbeamercolor{block body}{bg=gray!15!white}
\setbeamercolor{block title}{bg=gray!15!white}
\setbeamercolor{block body alerted}{bg=gray!15!white}
\setbeamercolor{block title alerted}{bg=gray!15!white}
\setbeamercolor{block body example}{bg=gray!15!white}
\setbeamercolor{block title example}{bg=gray!15!white}

\setbeamercolor{item}{fg=darkred}
\setbeamercolor{subitem}{fg=darkred}
\setbeamercolor{subsubitem}{fg=darkred}

\setbeamertemplate{title page}[default][rounded=true]

\beamertemplatenavigationsymbolsempty

\usepackage[utf8]{inputenc}
\usepackage[T1]{fontenc}
\usepackage{amsmath}
\usepackage{mathtools}
\usepackage{multicol}
\usepackage{graphicx}
\usepackage{siunitx}
\usepackage{pgfplots}
\usepackage[bottom]{footmisc}
\usepackage{amssymb}

\usepackage{float}

\usepackage[lining,scaled=0.9]{FiraMono}
\usepackage{raleway}
\usepackage{karnaugh-map}

\pgfplotsset{width=10cm,compat=1.9}
\graphicspath{{../figures/}}

%\titleformat*{\section}{\Large\bfseries}
%\titleformat*{\subsection}{\small\bfseries}
%\titlespacing*{\section}{0pt}{0mm}{0mm}
%\titlespacing*{\subsection}{0pt}{0mm}{0mm}
%\lstset{basicstyle=\ttfamily,breaklines=true}

\setlength{\parindent}{0pt}
\setlength{\parskip}{1ex}

\usepackage[usefamily=py,gobble=auto]{pythontex}
\usepackage[outputdir=./build]{minted}

\usepackage{circuitikz}

\def\normalcoord(#1){coordinate(#1)}
\def\showcoord(#1){node[circle, red, draw, inner sep=1pt,
	pin={[red, overlay, inner sep=0.5pt, font=\tiny, pin distance=0.1cm,
	pin edge={red, overlay}]45:#1}](#1){}}
\let\coord=\normalcoord
%\let\coord=\showcoord

\usepackage{minted}
\setminted{
	autogobble,
	fontsize=\tiny,
	tabsize=4,
	breaklines
}
\setpythontexfv{
	frame=lines,
	framesep=2mm,
	fontsize=\footnotesize,
	linenos,
	numbersep=6pt,
	tabsize=4,
	breaklines
}

\begin{pythontexcustomcode}[begin]{py}
from sympy import *
from sympy.parsing.latex import parse_latex as sympy_parse_latex

import matplotlib as mpl
import matplotlib.pyplot as plt
import numpy as np
import scipy.constants

from scipy.optimize import fsolve

def formatter(input):
	if type(input) == str:
		return input
	else:
		return latex(input)

pytex.formatter = formatter

def parse_latex(s):
	s = s.replace(r"\left","").replace(r"\right","")

	symbols_dict = {}
	globals_copy = globals().copy()
	for key, value in globals_copy.items():
		if str(type(value)) == "<class 'sympy.core.symbol.Symbol'>":
			symbols_dict[key] = value

	expr_str = str(sympy_parse_latex(s)).replace(r"{","").replace(r"}","")
	expr = parse_expr(expr_str, local_dict=symbols_dict)
	return expr

def SI(x, unit):
	x = float(x)
	sign = ''
	if x < 0:
		x = -x
		sign = '-'

	if x==0:
		exp = 0
		exp3 = 0
		x3 = 0
	else:
		exp = int(np.floor(np.log10(x)))
		exp3 = exp - (exp % 3)
		x3 = x / (10**exp3)
		x3 = round(x3, 3)
		if x3 == int(x3):
			x3 = int(x3)

	units = [
		r'\yocto ',
		r'\zepto ',
		r'\atto ',
		r'\femto ',
		r'\pico ',
		r'\nano ',
		r'\micro ',
		r'\milli ',
		'',
		r'\kilo ',
		r'\mega ',
		r'\giga ',
		r'\tera ',
		r'\peta ',
		r'\exa ',
		r'\zetta ',
		r'\yotta '
	]
	exp3_text = ''
	unit_text = ''
	if exp3 >= -24 and exp3 <= 24:
		unit_text = units[ exp3 // 3 + 8]
	else:
		exp3_text = "e" + str(exp3)

	return r"\SI{" + sign + str(x3) + exp3_text + r"}{" + unit_text + str(unit) + "}"

mpl.rc('text', usetex=True)
mpl.rc('font', family='serif', size=12.0)
mpl.rc('figure', figsize=[5, 3.5])
mpl.rc('lines', linewidth=1, color='k')
mpl.rc('savefig', bbox='tight', pad_inches=0.05)
mpl.rc('pgf', texsystem='pdflatex')
mpl.rc('legend', fancybox=False, edgecolor='k', framealpha=0.7)
mpl.rc('axes', titlesize='medium', labelsize='medium')
mpl.rc('grid', linestyle=':', alpha=0.5)

def reset_plot():
	plt.clf()
	plt.margins(x=0, y=0)
	plt.grid(True, which='major')
	plt.minorticks_on()
	plt.tick_params(axis='both', which='both', direction='in', top=True, right=True)

reset_plot()
\end{pythontexcustomcode}

\begin{pycode}
1+1
\end{pycode}
